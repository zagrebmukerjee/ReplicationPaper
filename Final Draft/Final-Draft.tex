% AER-Article.tex for AEA last revised 22 June 2011
\documentclass[]{AEA}

% The mathtime package uses a Times font instead of Computer Modern.
% Uncomment the line below if you wish to use the mathtime package:
%\usepackage[cmbold]{mathtime}
% Note that miktex, by default, configures the mathtime package to use commercial fonts
% which you may not have. If you would like to use mathtime but you are seeing error
% messages about missing fonts (mtex.pfb, mtsy.pfb, or rmtmi.pfb) then please see
% the technical support document at http://www.aeaweb.org/templates/technical_support.pdf
% for instructions on fixing this problem.

% Note: you may use either harvard or natbib (but not both) to provide a wider
% variety of citation commands than latex supports natively. See below.

% Uncomment the next line to use the natbib package with bibtex
\usepackage{natbib}

% Uncomment the next line to use the harvard package with bibtex
%\usepackage[abbr]{harvard}

% This command determines the leading (vertical space between lines) in draft mode
% with 1.5 corresponding to "double" spacing.
\draftSpacing{1.5}

% For Pandoc highlighting needs

% Pandoc citation processing


\usepackage{hyperref}

\begin{document}

\title{Brother, Can You Spare a Manufacturing Job? How Voters React to
Deindustrialization}
\shortTitle{Brother, Can You Spare a Manufacturing Job?}
% \author{Author1 and Author2\thanks{Surname1: affiliation1, address1, email1.
% Surname2: affiliation2, address2, email2. Acknowledgements}}


\author{
  Catherine Darin\\
  Zagreb Mukerjee\thanks{
  Darin: Harvard
University, \href{mailto:cdarin@hks.harvard.edu}{cdarin@hks.harvard.edu}.
  Mukerjee: Harvard
University, \href{mailto:zagrebmukerjee@fas.harvard.edu}{zagrebmukerjee@fas.harvard.edu}.
}
}

\date{\today}
\pubMonth{12}
\pubYear{2021}
\pubVolume{1}
\pubIssue{1}
\JEL{}
\Keywords{}

\begin{abstract}
What led to Donald Trump's surprising 2016 election victory? This paper
examines the potential contribution of deindustrialization. We use an
instrumental-variable approach and a covariate-balancing approach to
identify the effect of manufacturing job loss on change in Democratic
vote share. We find that Democratic vote share falls in counties with
high manufacturing job loss since 2004. Disaggregating this result by
race shows that the Democratic vote share fell where layoffs affected
white populations, and rose where layoffs affected nonwhites. This
suggests a racial component to how voters process economic hardship.
\end{abstract}


\maketitle

This paper analyzes the relationship between deindustrialization, race,
and the election of Donald Trump in 2016. Previous analysis has
frequently focused on racial animus, trade, or the conjunction of the
two (\cite{Autor20}; \cite{che16}; \cite{BR21}). We extend the analysis
of \cite{Baccini21}, one of the first papers to examine the effects of
deindustrialization in a localized way. Deindustrialization - the
transition of the U.S. economy away from manufacturing - has led to
widespread job loss and created knock-on adverse effects in many
formerly industrial areas. Along with \cite{Baccini21}, we find evidence
that the experience of deindustralization was associated with support
for Donald Trump, particularly among whites. However, we find that the
related effects in the 2016 election are not associated with recent
manufacturing job loss; rather, the opposite. Counties with
manufacturing job losses from 2012-2015 tended to become more
Democratic. We extend the analysis to the 2004-2015 period, to capture
the long-term effects of deindustrialization on a region. Doing so, we
find that regions experiencing more manufacturing job loss swung towards
Donald Trump in 2016, and that this occurred more when those losing jobs
were white.

Our analysis extends \cite{Baccini21} in several ways. First, we change
focus from a measure of gross losses in manufacturing jobs to net
losses. As discussed in \ref{datasec}, we believe this to be a more
accurate measure of deindustrialization. Second, we change the temporal
focus. Rather than focusing on manufacturing job loss in 2012-2015, we
extend our analysis back to 2004. This is an important change. As shown
XXXXXXXXXXXXXX FIGURE XXXXXXXXXXXXXXX, the bulk of U.S. manufacturing
job loss took place prior to 2012; in fact, the 2012-2015 period was
characterized by a slight rebound in manufacturing jobs. During this
rebound, counties that experienced the most pronounced manufacturing job
losses in prior years were also likely to experience the most pronounced
recovery XXXXXXXXXXXXXXX FIGURE XXXXXXXXXXXXX. Because of this v-shaped
recovery, focusing only on 2012-2015 job losses/gains paints a distorted
picture. Finally, we augment the instrumental-variable approach with one
based on covariate balancing, which produces similar results.

\section{Data and Methods} 
\label{datamethods}

\subsection{Data} 
\label{datasec}

Following \cite{Baccini21}, our unit of analysis is the county. This
allows us to capture not only the direct effects of deindustrialization
on laid-off manufacturing workers, but also on the local economy. The
primary outcome variable is the county's change in Democratic vote share
between 2012 and 2016. We obtain this from XXXXXXXXX

We want to measure the causal relationship between deindustrialization
and the change in vote share. As a measure of deindustrialization, we
use the loss of manufacturing jobs as a share of the total
beginning-of-period employment in each county. Suppose at the beginning
of our period there were \(2000\) manufacturing workers in a county, and
\(8000\) non-manufacturing workers. At the end there are only \(1500\)
manufacturing workers. This is a loss of \(500/10000 = 5\%.\)

In this choice we diverge from \cite{Baccini21}, which uses
\textbf{gross} manufacturing job losses. For example, if a county lost
\(450\) manufacturing jobs from 2012-2015 and also gained \(400\)
manufacturing jobs over the same period, this would be \(450\)
\textbf{gross} job losses, but only \(50\) \textbf{net} job losses. The
\textbf{gross} measure captures several dynamics unrelated to
deindustrialization - for example, seasonal unemployment in a
food-manufacturing region, or workers moving between jobs. Thus we
believe the \textbf{net} job losses over a period are the more accurate
measure.

Data on job gains and losses are obtained, using an API, from the Census
Bureau's Quarterly Workforce Indicators (\cite{QWI}), which contains
(among other things) information about employment by industry and
county. These statistics are further disaggregated by race and
ethnicity. The Census Bureau obtains this data from a combination of
sources, such as administrative tax data and the U.S. Census.

XXXXXXXXXXXXXXXX sources of other data

\subsection{Methods} 
\label{methodssec}

We conduct two related tests of our hypothesis. Our first test involves
an instrumental-variable approach. A regression of change in Democratic
vote share on manufacturing job losses risks endogeneity, in case
counties with manufacturing job losses were otherwise predisposed to
turn towards Trump (a singular candidate, after all). To mitigate these
risks, \cite{Baccini21} uses a Bartik instrument, which we adapt (see
\cite{Bartik91}). This instrument essentially uses the cross-county
distribution of manufacturing employment as a source of exogenous
variation.

\[
\begin{aligned}
b_{j,c} &=& \frac{\text{Manufacturing Employment}_{j,c} \text{ at } t_0}{\text{Total Employment}_c \text{ at }t_0}  \\
&&  * \frac{\text{National Manufacturing Job Change}_{j,c} }{\text{Total National Employment at }t_0} \\
\end{aligned}
\]

Using this instrument, we then conduct a two stage regression. First we
estimate manufacturing job loss with the Bartik Instrument and a set of
county-level controls; then we use the estimated values of job losses to
predict Democratic vote share. In these regressions we control for
unemployment, service layoffs, the share of college-educated voters, and
the share of male voters. For some estimates, we also control for the
white share of the population. These controls are similar to those of
\cite{Baccini21}.

Our second approach involves attempting to balance the covariates of the
deindustrialization treatment. We use the same covariates as are used in
controls above, as well as manufacturing as share of the population and
total population. We apply the method of Covariate Balancing Propensity
Scores, as developed by \cite{Imai14}; this method generates weights to
reduce the correlation between covariates and treatment, while
addressing several of the issues with propensity score weights. Given
covariate-balancing weights, we then conduct a single-stage weighted
regression of change in Democratic vote share on manufacturing job loss,
using the same controls as above.

XXXXXXXXXXXXXXX BALANCE FIGURE??? XXXXXXXXXXXXXXXXXXXXXXXXXXX

\subsection{Results}

\subsection{Conclusion}

If we expand our measure of net manufacturing job losses to include the
2004 to 2011 period - which covers most of the period when the U.S.
manufacturing economy was hit most intensely (See Figure 1) -- we get a
result that corroborates the core intuition of Baccini and Weymouth (See
Table 4). That is, we find that net white manufacturing job losses
decreased Democratic vote share, while net non-white manufacturing job
losses increased Democratic vote share.

It is important to highlight the substantive importance of the
\emph{timing} of manufacturing layoffs on political outcomes. Whereas
Baccini and Weymouth's analysis suggests that economic changes lead to
immediate political consequences, our analysis shows that economic
changes can have lagged and variable effects on political outcomes. The
mechanisms of these lagged effects merit further study.

{[}Note: to demonstrate our point about \textbf{variable} effects, In
our next iteration of the analysis, we will show that manufacturing job
losses had little explanatory effect on changes in Democratic vote share
in other elections (2012 and 2020).{]}

Sample figure:

American Economic Review Pointers:

\begin{itemize}
\item Do not use an "Introduction" heading. Begin your introductory material
before the first section heading.

\item Avoid style markup (except sparingly for emphasis).

\item Avoid using explicit vertical or horizontal space.

\item Captions are short and go below figures but above tables.

\item The tablenotes or figurenotes environments may be used below tables
or figures, respectively, as demonstrated below.

\item If you have difficulties with the mathtime package, adjust the package
options appropriately for your platform. If you can't get it to work, just
remove the package or see our technical support document online (please
refer to the author instructions).

\item If you are using an appendix, it goes last, after the bibliography.
Use regular section headings to make the appendix headings.

\item If you are not using an appendix, you may delete the appendix command
and sample appendix section heading.

\item Either the natbib package or the harvard package may be used with bibtex.
To include one of these packages, uncomment the appropriate usepackage command
above. Note: you can't use both packages at once or compile-time errors will result.

\end{itemize}

\section{First Section in Body}

Sample figure:

\begin{figure}

\caption{Caption for figure below.}
\begin{figurenotes}
Figure notes without optional leadin.
\end{figurenotes}
\begin{figurenotes}[Source]
Figure notes with optional leadin (Source, in this case).
\end{figurenotes}
\end{figure}

Sample table:

\begin{table}
\caption{Caption for table above.}

\begin{tabular}{lll}
& Heading 1 & Heading 2 \\
Row 1 & 1 & 2 \\
Row 2 & 3 & 4%
\end{tabular}
\begin{tablenotes}
Table notes environment without optional leadin.
\end{tablenotes}
\begin{tablenotes}[Source]
Table notes environment with optional leadin (Source, in this case).
\end{tablenotes}
\end{table}

References here (manual or bibTeX). If you are using bibTeX, add your
bib file name in place of BibFile in the bibliography command. \% Remove
or comment out the next two lines if you are not using bibtex.

\bibliographystyle{aea}
\bibliography{references}

\end{document}
