% AER-Article.tex for AEA last revised 22 June 2011
\documentclass[]{AEA}

% The mathtime package uses a Times font instead of Computer Modern.
% Uncomment the line below if you wish to use the mathtime package:
%\usepackage[cmbold]{mathtime}
% Note that miktex, by default, configures the mathtime package to use commercial fonts
% which you may not have. If you would like to use mathtime but you are seeing error
% messages about missing fonts (mtex.pfb, mtsy.pfb, or rmtmi.pfb) then please see
% the technical support document at http://www.aeaweb.org/templates/technical_support.pdf
% for instructions on fixing this problem.

% Note: you may use either harvard or natbib (but not both) to provide a wider
% variety of citation commands than latex supports natively. See below.

% Uncomment the next line to use the natbib package with bibtex
\usepackage{natbib}

% Uncomment the next line to use the harvard package with bibtex
%\usepackage[abbr]{harvard}

% This command determines the leading (vertical space between lines) in draft mode
% with 1.5 corresponding to "double" spacing.
\draftSpacing{1.5}

% For Pandoc highlighting needs

% Pandoc citation processing


\usepackage{hyperref}

\begin{document}

\title{Brother, Can You Spare a Manufacturing Job? How Voters React to
Deindustrialization}

% \author{Author1 and Author2\thanks{Surname1: affiliation1, address1, email1.
% Surname2: affiliation2, address2, email2. Acknowledgements}}


\author{
  Catherine Darin\\
  Zagreb Mukerjee\thanks{
  Darin: Harvard University, \href{mailto:cdarin@hks.harvard.edu}{cdarin@hks.harvard.edu}.
  Mukergee: Harvard University, \href{mailto:zagrebmukerjee@fas.harvard.edu}{zagrebmukerjee@fas.harvard.edu}.
}
}

\date{\today}
\pubMonth{12}
\pubYear{2021}
\pubVolume{1}
\pubIssue{1}
\JEL{, }
\Keywords{, }

\begin{abstract}
What led to Donald Trump's surprising 2016 election victory? This paper
examines the potential contribution of deindustrialization. Counties
differ by exposure to manufacturing, which we use in an
instrumental-variable approach to identify the effect of manufacturing
job loss on change in Democratic vote share. We find that Democratic
vote share falls in counties with high manufacturing job loss since
2004. Disaggregating this result by race shows that the Democratic vote
share fell where layoffs affected white populations, and rose where
layoffs affected nonwhites. This suggests a racial component to how
voters process economic hardship.
\end{abstract}


\maketitle

\section{Introduction}

We replicate Baccini and Wemouth (2021), which analyzes the effect of
gross manufacturing job losses from 2012 to 2015 on the change in
Democratic vote share in the 2016 presidential election.

We extend Baccini and Weymouth's analysis by changing the quantity of
interest from \textbf{gross} manufacturing job losses to \textbf{net}
manufacturing job losses. For example, in Baccini and Weymouth's model,
if a county lost 400 manufacturing jobs from 2012-2015 and also gained
400 manufacturing jobs over the same period, they would count 400
overall layoffs. Using our measure of \emph{net} manufacturing job
losses, we would consider this example a 0 net change in manufacturing
employment. Given that job destruction and creation is a natural dynamic
within the economy (i.e.~seasonal employment, creative destruction), we
think that \emph{net} job losses is a more meaningful measure.

Using \emph{net} manufacturing job losses as our outcome, we replicate
Baccini and Weymouth's main model, which uses 2012-2015 manufacturing
(gross) job losses to explain the change in Democratic vote share in
2016. Using their specifications, we find totally opposite results -
that a \emph{net} increase in white manufacturing jobs \emph{increased}
Democratic vote share, while a \emph{net} increase in non-white
manufacturing jobs \emph{decreased} Democratic vote share.

However, we don't think this is capturing the important dynamic at play.
From 2012 to 2015, the national economy experienced a net increase in
manufacturing jobs following the 2008 recession (See Figure 1). During
this time of recovery, counties that experienced the most pronounced
manufacturing job losses in prior years were also likely to experience
the most pronounced recovery. Because of this v-shaped recovery,
focusing only on 2012-2015 job losses/gains paints a distorted picture.

\section{First Section in Body}

Sample figure:

\begin{figure}





\caption{Caption for figure below.}
\begin{figurenotes}
Figure notes without optional leadin.
\end{figurenotes}
\begin{figurenotes}[Source]
Figure notes with optional leadin (Source, in this case).
\end{figurenotes}
\end{figure}

If we expand our measure of net manufacturing job losses to include the
2004 to 2011 period - which covers most of the period when the U.S.
manufacturing economy was hit most intensely (See Figure 1) -- we get a
result that corroborates the core intuition of Baccini and Weymouth (See
Table 4). That is, we find that net white manufacturing job losses
decreased Democratic vote share, while net non-white manufacturing job
losses increased Democratic vote share.

It is important to highlight the substantive importance of the
\emph{timing} of manufacturing layoffs on political outcomes. Whereas
Baccini and Weymouth's analysis suggests that economic changes lead to
immediate political consequences, our analysis shows that economic
changes can have lagged and variable effects on political outcomes. The
mechanisms of these lagged effects merit further study.

{[}Note: to demonstrate our point about \textbf{variable} effects, In
our next iteration of the analysis, we will show that manufacturing job
losses had little explanatory effect on changes in Democratic vote share
in other elections (2012 and 2020).{]}

Sample figure:

American Economic Review Pointers:

\begin{itemize}
\item Do not use an "Introduction" heading. Begin your introductory material
before the first section heading.

\item Avoid style markup (except sparingly for emphasis).

\item Avoid using explicit vertical or horizontal space.

\item Captions are short and go below figures but above tables.

\item The tablenotes or figurenotes environments may be used below tables
or figures, respectively, as demonstrated below.

\item If you have difficulties with the mathtime package, adjust the package
options appropriately for your platform. If you can't get it to work, just
remove the package or see our technical support document online (please
refer to the author instructions).

\item If you are using an appendix, it goes last, after the bibliography.
Use regular section headings to make the appendix headings.

\item If you are not using an appendix, you may delete the appendix command
and sample appendix section heading.

\item Either the natbib package or the harvard package may be used with bibtex.
To include one of these packages, uncomment the appropriate usepackage command
above. Note: you can't use both packages at once or compile-time errors will result.

\end{itemize}

\section{First Section in Body}

Sample figure:

\begin{figure}
Figure here.

\caption{Caption for figure below.}
\begin{figurenotes}
Figure notes without optional leadin.
\end{figurenotes}
\begin{figurenotes}[Source]
Figure notes with optional leadin (Source, in this case).
\end{figurenotes}
\end{figure}

Sample table:

\begin{table}
\caption{Caption for table above.}

\begin{tabular}{lll}
& Heading 1 & Heading 2 \\
Row 1 & 1 & 2 \\
Row 2 & 3 & 4%
\end{tabular}
\begin{tablenotes}
Table notes environment without optional leadin.
\end{tablenotes}
\begin{tablenotes}[Source]
Table notes environment with optional leadin (Source, in this case).
\end{tablenotes}
\end{table}

References here (manual or bibTeX). If you are using bibTeX, add your
bib file name in place of BibFile in the bibliography command. \% Remove
or comment out the next two lines if you are not using bibtex.

\bibliographystyle{aea}
\bibliography{references}

\% The appendix command is issued once, prior to all appendices, if any.
\appendix

\section{Mathematical Appendix}

\end{document}

